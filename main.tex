\documentclass[10pt, letterpaper]{article}

% Packages:
\usepackage[
    ignoreheadfoot, % set margins without considering header and footer
    top=2 cm, % seperation between body and page edge from the top
    bottom=2 cm, % seperation between body and page edge from the bottom
    left=2 cm, % seperation between body and page edge from the left
    right=2 cm, % seperation between body and page edge from the right
    footskip=1.0 cm, % seperation between body and footer
    % showframe % for debugging 
]{geometry} % for adjusting page geometry
\usepackage{titlesec} % for customizing section titles
\usepackage{tabularx} % for making tables with fixed width columns
\usepackage{array} % tabularx requires this
\usepackage[dvipsnames]{xcolor} % for coloring text
\definecolor{primaryColor}{RGB}{0, 79, 144} % define primary color
\usepackage{enumitem} % for customizing lists
\usepackage{fontawesome5} % for using icons
\usepackage{amsmath} % for math
\usepackage[
    pdftitle={Haozhou's CV},
    pdfauthor={Haozhou Wang},
    pdfcreator={LaTeX with RenderCV},
    colorlinks=true,
    urlcolor=primaryColor
]{hyperref} % for links, metadata and bookmarks
\usepackage[pscoord]{eso-pic} % for floating text on the page
\usepackage{calc} % for calculating lengths
\usepackage{bookmark} % for bookmarks
\usepackage{lastpage} % for getting the total number of pages
\usepackage{changepage} % for one column entries (adjustwidth environment)
\usepackage{paracol} % for two and three column entries
\usepackage{ifthen} % for conditional statements
\usepackage{needspace} % for avoiding page brake right after the section title
\usepackage{iftex} % check if engine is pdflatex, xetex or luatex

%%%%%%%%%%%%%%%%%%%%
% 右上角角标日期设置
%%%%%%%%%%%%%%%%%%%%
\usepackage[en-US]{datetime2}
\DTMsetup{useregional}  % 设置日期格式为美国英语

%%%%%%%%%%%%%%%%%%%%%
% 自动参考文献列表设置
%%%%%%%%%%%%%%%%%%%%%
\usepackage[UTF8]{ctex}  % 加载 ctex 宏包,支持中文
\usepackage[
    backend=biber,       % 使用 biber 后端
    style=authoryear,    % Author-Year 引用样式
    dashed=false,        % 禁止重复作者用横线代替
    doi=true,            % 显示 DOI
    url=false,           % 不显示 URL(除非没有 DOI)
    maxbibnames=10,      % 显示所有作者
    giveninits=true,     % 缩写名字
    sorting=ydnt,        % 按年份降序排序(新->旧)
    date=year,           % 默认只显示年份
]{biblatex}

% 加粗和下划线自己的名字 
\DeclareNameFormat{highlight}{%
  \nameparts{#1}% 确保姓名解析
  \ifboolexpr{
    test {\ifdefstring{\namepartfamily}{Wang}} 
    and 
    test {\ifdefstring{\namepartgiven}{Haozhou}}
  }
    {\textbf{\underline{\namepartfamily\addcomma\space\namepartgiveni}} }   % 加粗且缩写
    {\namepartfamily\addcomma\space\namepartgiveni}%           % 正常缩写
  % 修复方案:完整保留多作者分隔逻辑
  \ifthenelse{\value{listcount}<\value{liststop}}
    {\addcomma\space}
    {}%
  \usebibmacro{name:andothers} % 关键:保留多作者分隔逻辑
}
% 应用上述逻辑
\DeclareNameAlias{author}{highlight}
\DeclareNameAlias{editor}{highlight}
\DeclareNameAlias{translator}{highlight}

% 标题不添加引号
\DeclareFieldFormat[article]{title}{#1}
\DeclareFieldFormat[report]{title}{#1}
\DeclareFieldFormat[incollection]{title}{#1}

% 加载 BibLaTeX 文件% 自定义分类标题格式(可选)
\addbibresource{publications.bib} 

% 定义计数器,统计发表论文数量
\newcounter{bookchapcount}
\newcounter{journalcount}
\newcounter{confcount}

% 在打印文献前统计数量
\AtDataInput{%
  \ifentrytype{incollection}{\stepcounter{bookchapcount}}{}%
  \ifentrytype{article}{\stepcounter{journalcount}}{}%
  \ifentrytype{report}{\stepcounter{confcount}}{}%
}

% Ensure that generate pdf is machine readable/ATS parsable:
\ifPDFTeX
    \input{glyphtounicode}
    \pdfgentounicode=1
    % \usepackage[T1]{fontenc} % this breaks sb2nov
    \usepackage[utf8]{inputenc}
    \usepackage{lmodern}
\fi

% Some settings:
\AtBeginEnvironment{adjustwidth}{\partopsep0pt} % remove space before adjustwidth environment
\pagestyle{empty} % no header or footer
\setcounter{secnumdepth}{0} % no section numbering
\setlength{\parindent}{0pt} % no indentation
\setlength{\topskip}{0pt} % no top skip
\setlength{\columnsep}{0cm} % set column seperation
\makeatletter
\let\ps@customFooterStyle\ps@plain % Copy the plain style to customFooterStyle
\patchcmd{\ps@customFooterStyle}{\thepage}{
    \color{gray}\textit{\small Haozhou Wang - Page \thepage{} of \pageref*{LastPage}}
}{}{} % replace number by desired string
\makeatother
\pagestyle{customFooterStyle}

\titleformat{\section}{\needspace{4\baselineskip}\bfseries\large}{}{0pt}{}[\vspace{1pt}\titlerule]

\titlespacing{\section}{
    % left space:
    -1pt
}{
    % top space:
    0.3 cm
}{
    % bottom space:
    0.2 cm
} % section title spacing

\renewcommand\labelitemi{$\circ$} % custom bullet points
\newenvironment{highlights}{
    \begin{itemize}[
        topsep=0.10 cm,
        parsep=0.10 cm,
        partopsep=0pt,
        itemsep=0pt,
        leftmargin=0.4 cm + 10pt
    ]
}{
    \end{itemize}
} % new environment for highlights

\newenvironment{highlightsforbulletentries}{
    \begin{itemize}[
        topsep=0.10 cm,
        parsep=0.10 cm,
        partopsep=0pt,
        itemsep=0pt,
        leftmargin=10pt
    ]
}{
    \end{itemize}
} % new environment for highlights for bullet entries


\newenvironment{onecolentry}{
    \begin{adjustwidth}{
        0.2 cm + 0.00001 cm
    }{
        0.2 cm + 0.00001 cm
    }
}{
    \end{adjustwidth}
} % new environment for one column entries

\newenvironment{twocolentry}[2][]{
    \onecolentry
    \def\secondColumn{#2}
    \setcolumnwidth{\fill, 4.5 cm}
    \begin{paracol}{2}
}{
    \switchcolumn \raggedleft \secondColumn
    \end{paracol}
    \endonecolentry
} % new environment for two column entries

\newenvironment{header}{
    \setlength{\topsep}{0pt}\par\kern\topsep\centering\linespread{1.5}
}{
    \par\kern\topsep
} % new environment for the header

%%%%%%%%%%%%%%%%%%%%
% 右上角角标日期设置
%%%%%%%%%%%%%%%%%%%%

% 自定义日期格式为 "Sept. 2024" 样式
\newcommand{\customdate}{%
  \DTMenglishmonthname{\month} \number\year%
}

\newcommand{\placelastupdatedtext}{% \placetextbox{<horizontal pos>}{<vertical pos>}{<stuff>}
  \AddToShipoutPictureFG*{% Add <stuff> to current page foreground
    \put(
        \LenToUnit{\paperwidth-2 cm-0.2 cm+0.05cm},
        \LenToUnit{\paperheight-1.0 cm}
    ){\vtop{{\null}\makebox[0pt][c]{
        % \small\color{gray}\textit{Last updated in September 2024}\hspace{\widthof{Last updated in September 2024}}
        \small\color{gray}\textit{Last updated in \customdate}\hspace{\widthof{Last updated in \customdate}}
    }}}%
  }%
}%

% save the original href command in a new command:
\let\hrefWithoutArrow\href

% new command for external links:
\renewcommand{\href}[2]{\hrefWithoutArrow{#1}{\ifthenelse{\equal{#2}{}}{ }{#2 }\raisebox{.15ex}{\footnotesize \faExternalLink*}}}


\begin{document}
    \newcommand{\AND}{\unskip
        \cleaders\copy\ANDbox\hskip\wd\ANDbox
        \ignorespaces
    }
    \newsavebox\ANDbox
    \sbox\ANDbox{}

    \placelastupdatedtext
    \begin{header}
        \textbf{\fontsize{24 pt}{24 pt}\selectfont Haozhou Wang}

        \vspace{0.3 cm}

        Project Research Assistant, Laboratory of Field Phenomics, 
        
        Graduate School of Agricultural and Life Sciences, The University of Tokyo.

        \normalsize
        % \mbox{{\color{black}\footnotesize\faMapMarker*}\hspace*{0.13cm}Tokyo, Japan}%
        % \kern 0.25 cm%
        % \AND%
        % \kern 0.25 cm%
        \mbox{\hrefWithoutArrow{mailto:haozhou-wang@outlook.com}{\color{black}{\footnotesize\faEnvelope[regular]}\hspace*{0.13cm}haozhou-wang@outlook.com}}%
        \kern 0.25 cm%
        % \AND%
        % \kern 0.25 cm%
        % \mbox{\hrefWithoutArrow{tel:+90-541-999-99-99}{\color{black}{\footnotesize\faPhone*}\hspace*{0.13cm}0541 999 99 99}}%
        % \kern 0.25 cm%
        \AND%
        \kern 0.25 cm%
        \mbox{\hrefWithoutArrow{https://www.haozhou.wang/}{\color{black}{\footnotesize\faLink}\hspace*{0.13cm}haozhou.wang}}%
        \kern 0.25 cm%
        % \AND%
        % \kern 0.25 cm%
        % \mbox{\hrefWithoutArrow{https://www.linkedin.com/in/haozhouwang/}{\color{black}{\footnotesize\faLinkedinIn}\hspace*{0.13cm}haozhouwang}}%
        % \kern 0.25 cm%
        \AND%
        \kern 0.25 cm%
        \mbox{\hrefWithoutArrow{https://github.com/HowcanoeWang}{\color{black}{\footnotesize\faGithub}\hspace*{0.13cm}HowcanoeWang}}%
        \kern 0.25 cm%
        \AND%
        \kern 0.25 cm%
        \mbox{\hrefWithoutArrow{https://scholar.google.com/citations?user=CnTTa3EAAAAJ}{\color{black}{\footnotesize\faGraduationCap}\hspace*{0.13cm}286 citations, h-index 8}}%
    \end{header}

    \vspace{0.3 cm - 0.3 cm}

    \section{Research Interests}

        \begin{onecolentry}
            \begin{highlightsforbulletentries}

            \item High-throughput plant 3D phenotyping.

            \item Digital twin virtual plant model and multi-sensory data fusion.

            \item Open-source agricultural phenotyping tool and dataset development.

            \end{highlightsforbulletentries}
        \end{onecolentry}

    \section{Professional Positions}
        
        \begin{twocolentry}{
            \textit{Tokyo, Japan}    
            
            \textit{Oct 2023 – present}}
            \textbf{Project Research Assistant}
            
            \textit{The University of Tokyo}
        \end{twocolentry}

        \vspace{0.10 cm}
        \begin{onecolentry}
            \begin{highlights}
                \item Focused on 3D-based plant phenotyping by multi-scale data fusion.
            \end{highlights}
        \end{onecolentry}


        % \vspace{0.2 cm}

        % \begin{twocolentry}{
        % \textit{Beijing, China}    
            
        % \textit{May 2016 – Aug 2016}}
        %     \textbf{Research Student Intern}
            
        %     \textit{Chinese Academy of Forestry}
        % \end{twocolentry}

        % \vspace{0.10 cm}
        % \begin{onecolentry}
        %     \begin{highlights}
        %         \item Developed software and website for UAV image analysis, named UAV-HiRAP (\href{https://www.uav-hirap.org}{www.uav-hirap.org}).
        %         \item Extended to an oral presentation on the 12th International Congress of Ecological.
        %         \item One publication on top journal Agricultural and Forest Meteorology.
        %     \end{highlights}
        % \end{onecolentry}


    \section{Education}
        
        \begin{twocolentry}{
            \textit{Oct. 2020 – Sept. 2023}}
            \textbf{The University of Tokyo}

            \textit{Doctor in Agricultural Science}
        \end{twocolentry}

        \vspace{0.10 cm}
        \begin{onecolentry}
            \underline{Thesis title:} Studies on 3D-based plant phenotyping by multi-scale data fusion.
            % \begin{highlights}
            %     \item GPA: 3.9/4.0 (\href{https://example.com}{a link to somewhere})
            %     \item 
            % \end{highlights}
        \end{onecolentry}

        % \vspace{0.20 cm}

        % \begin{twocolentry}{
        %     \textit{Sept. 2017 – Dec. 2019}}
        %     \textbf{The University of New Brunswick}

        %     \textit{Master of Science in Forestry}
        % \end{twocolentry}

        % \vspace{0.10 cm}
        % \begin{onecolentry}
        %     \begin{highlights}
        %         % \item GPA: 3.9/4.0 (\href{https://example.com}{a link to somewhere})
        %         \item \textbf{Thesis Title:} Estimating Forest Attributes from Spherical Images.
        %     \end{highlights}
        % \end{onecolentry}

        % \vspace{0.20 cm}

        % \begin{twocolentry}{
        %     \textit{Sept. 2013 – Jun. 2017}}
        %     \textbf{The Nanjing Forestry University}

        %     \textit{Bachelor of Science in Ecology}
        % \end{twocolentry}

        % \vspace{0.10 cm}
        % \begin{onecolentry}
        %     \begin{highlights}
        %         % \item GPA: 3.9/4.0 (\href{https://example.com}{a link to somewhere})
        %         \item \textbf{Thesis Title:} Extracting DBH Measurements from RGB Photo Images.
        %     \end{highlights}
        % \end{onecolentry}

    
    \section{Featured Projects and Publications}
        
        \begin{twocolentry}{
            \textit{\href{https://github.com/UTokyo-FieldPhenomics-Lab/EasyIDP}{github repo}}}
            \textbf{EasyIDP}
        \end{twocolentry}
        
        \vspace{0.10 cm}
        \begin{onecolentry}
            \textit{A handy tool for dealing with region of interest (ROI) on the image reconstruction (Metashape \& Pix4D) outputs, mainly in agriculture applications.}

            \begin{highlights}
                \item \textbf{Publication}: \fullcite{wang_easyidp_2021}; \textbf{Citations}: 35; \textbf{Github stars}: 47.
                % \item \textbf{Tools Used}: Python, PyPi, Readthedocs; .
            \end{highlights}
        \end{onecolentry}


        \vspace{0.2 cm}

        \begin{twocolentry}{
            \textit{\href{https://www.uav-hirap.org}{uav-hirap.org}}}
            \textbf{UAV-HiRAP platform}
        \end{twocolentry}

        \vspace{0.10 cm}
        \begin{onecolentry}
            \textit{Unmanned Aerial Vehicles - High Resolution imagery Analysis Platform (UAV-HiRAP), is an open-source, web-based platform which provides service for image classification (maintenance suspended).}
            
            \begin{highlights}
                \item \textbf{Publication}: \fullcite{wang_landscape_2019};\textbf{Citations}: 40.
                % \item \textbf{Tools Used}: Python, Flask, Bootstrap, Nginx; 
            \end{highlights}
        \end{onecolentry}


        \vspace{0.2 cm}

        \begin{twocolentry}{
            \textit{\href{https://github.com/UTokyo-FieldPhenomics-Lab/UAVbroccoli}{github repo}}}
            \textbf{Broccoli harvest date prediction}
        \end{twocolentry}

        \vspace{0.10 cm}
        \begin{onecolentry}
            \begin{highlights}
                \item \textbf{Publication}: \fullcite{wang_dronebased_2023};\textbf{Citations}: 7.
                \item Achievement reported by \href{https://www.eurekalert.org/news-releases/1003639}{EurekAlert!}, \href{https://www.nikkei.com/article/DGXZRSP661800_Y3A900C2000000/}{日本日経新聞}, and \href{https://www.bjnews.com.cn/detail/1698978347129712.html}{新京报}
            \end{highlights}
        \end{onecolentry}

    
    \section{Publications}
        \nocite{*}

        \vspace{0.10 cm}
        \textbf{\textit{Book Chapters}} (\thebookchapcount\ entries)
        \printbibliography[type=incollection, heading=none] % 重置编号

        \vspace{0.10 cm}
        \textbf{\textit{Journal articles}} (\thejournalcount\ entries)
        \printbibliography[type=article, heading=none] % 重置编号

        \vspace{0.10 cm}
        \textbf{\textit{Conference proceedings}} (\theconfcount\ entries)
        \printbibliography[type=report, heading=none] % 重置编号
        
        % \begin{samepage}
        %     \begin{twocolentry}{
        %         Jan 2004
        %     }
        %         \textbf{3D Finite Element Analysis of No-Insulation Coils}

        %         \vspace{0.10 cm}

        %         \mbox{Frodo Baggins}, \mbox{\textbf{\textit{John Doe}}}, \mbox{Samwise Gamgee}
        %     \end{twocolentry}


        %     \vspace{0.10 cm}

        %     \begin{onecolentry}
        % \href{https://doi.org/10.1109/TASC.2023.3340648}{10.1109/TASC.2023.3340648}
        %     \end{onecolentry}
        % \end{samepage}





    
    % \section{Technologies}



        
    %     \begin{onecolentry}
    %         \textbf{Languages:} C++, C, Java, Objective-C, C\#, SQL, JavaScript
    %     \end{onecolentry}

    %     \vspace{0.2 cm}

    %     \begin{onecolentry}
    %         \textbf{Technologies:} .NET, Microsoft SQL Server, XCode, Interface Builder
    %     \end{onecolentry}


    

\end{document}